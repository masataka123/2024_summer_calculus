\documentclass[dvipdfmx,a4paper,11pt]{article}
\usepackage[utf8]{inputenc}
%\usepackage[dvipdfmx]{hyperref} %リンクを有効にする
\usepackage{url} %同上
\usepackage{amsmath,amssymb} %もちろん
\usepackage{amsfonts,amsthm,mathtools} %もちろん
\usepackage{braket,physics} %あると便利なやつ
\usepackage{bm} %ラプラシアンで使った
\usepackage[top=15truemm,bottom=30truemm,left=25truemm,right=25truemm]{geometry} %余白設定
\usepackage{latexsym} %ごくたまに必要になる
\renewcommand{\kanjifamilydefault}{\gtdefault}
\usepackage{otf} %宗教上の理由でmin10が嫌いなので
\usepackage{showkeys}\renewcommand*{\showkeyslabelformat}[1]{\fbox{\parbox{2cm}{ \normalfont\tiny\sffamily#1\vspace{6mm}}}}
\usepackage[driverfallback=dvipdfm]{hyperref}


\usepackage[all]{xy}
\usepackage{amsthm,amsmath,amssymb,comment}
\usepackage{amsmath}    % \UTF{00E6}\UTF{0095}°\UTF{00E5}\UTF{00AD}\UTF{00A6}\UTF{00E7}\UTF{0094}¨
\usepackage{amssymb}  
\usepackage{color}
\usepackage{amscd}
\usepackage{amsthm}  
\usepackage{wrapfig}
\usepackage{comment}	
\usepackage{graphicx}
\usepackage{setspace}
\usepackage{pxrubrica}
\usepackage{enumitem}
\usepackage{mathrsfs} 

\setstretch{1.2}


\newcommand{\R}{\mathbb{R}}
\newcommand{\Z}{\mathbb{Z}}
\newcommand{\Q}{\mathbb{Q}} 
\newcommand{\N}{\mathbb{N}}
\newcommand{\C}{\mathbb{C}} 
\newcommand{\Sin}{\text{Sin}^{-1}} 
\newcommand{\Cos}{\text{Cos}^{-1}} 
\newcommand{\Tan}{\text{Tan}^{-1}} 
\newcommand{\invsin}{\text{Sin}^{-1}} 
\newcommand{\invcos}{\text{Cos}^{-1}} 
\newcommand{\invtan}{\text{Tan}^{-1}} 
\newcommand{\Area}{\text{Area}}
\newcommand{\vol}{\text{Vol}}
\newcommand{\maru}[1]{\raise0.2ex\hbox{\textcircled{\tiny{#1}}}}
\newcommand{\sgn}{{\rm sgn}}
%\newcommand{\rank}{{\rm rank}}



   %当然のようにやる.
\allowdisplaybreaks[4]
   %もちろん.
%\title{第1回. 多変数の連続写像 (岩井雅崇, 2020/10/06)}
%\author{岩井雅崇}
%\date{2020/10/06}
%ここまで今回の記事関係ない
\usepackage{tcolorbox}
\tcbuselibrary{breakable, skins, theorems}

\theoremstyle{definition}
\newtheorem{thm}{定理}
\newtheorem{lem}[thm]{補題}
\newtheorem{prop}[thm]{命題}
\newtheorem{cor}[thm]{系}
\newtheorem{claim}[thm]{主張}
\newtheorem{dfn}[thm]{定義}
\newtheorem{rem}[thm]{注意}
\newtheorem{exa}[thm]{例}
\newtheorem{conj}[thm]{予想}
\newtheorem{prob}[thm]{問題}
\newtheorem{rema}[thm]{補足}

\DeclareMathOperator{\Ric}{Ric}
\DeclareMathOperator{\Vol}{Vol}
 \newcommand{\pdrv}[2]{\frac{\partial #1}{\partial #2}}
 \newcommand{\drv}[2]{\frac{d #1}{d#2}}
  \newcommand{\ppdrv}[3]{\frac{\partial #1}{\partial #2 \partial #3}}


%ここから本文.
\begin{document}
\pagestyle{empty}



\begin{center}
{\Large 演習問題 2024年6月6日(木)} \\

%\vspace{5pt}
%{ \large 提出締め切り 2024年月日(火) 15時10分00秒 (日本標準時刻)}
\end{center}

%\vspace{2pt}
%\begin{flushleft}
%{ \large \underline{学籍番号: \hspace{4cm} 名前  \hspace{8cm} } }
%\end{flushleft}

\begin{center}
 {\large 下の問題を解け. なお解答は配布した解答用紙に解答すること.}
  \end{center}
 ただし解答に関しては答えのみならず, 答えを導出する過程をきちんと記すこと. 
 また解答用紙は1人1枚以上提出すること.
  
  \vspace{11pt}
 問題1.  次の極限の値を求めよ.
 
 \begin{enumerate}
   \setlength{\parskip}{0cm} % 段落間
  \setlength{\itemsep}{0cm} % 項目間
 \item $\lim_{n \to \infty } \frac{n}{n^2 +1}$
 \item $\lim_{n \to \infty } \sum_{k=1}^{n} \frac{1}{5^k}$
 \item $\lim_{n \to \infty} \left(1 - \frac{1}{n}\right)^n$
 \end{enumerate}
 
\vspace{5pt}
 問題2. 次の関数の微分を求めよ.

\begin{enumerate}
   \setlength{\parskip}{0cm} % 段落間
  \setlength{\itemsep}{0cm} % 項目間
    \item $\frac{x}{x^2 +1}$ 
    
    \item $ \sqrt{x^2 + 1}$
    
    \item $  \log (2x^2 + 3x)$ 
\end{enumerate}
 
 \vspace{5pt}
 問題3. 次の問いに答えよ.

\begin{enumerate}
   \setlength{\parskip}{0cm} % 段落間
  \setlength{\itemsep}{0cm} % 項目間
    \item 関数 $f(x) = x^3 - 6x^2 + 9x + 1$ のグラフをかけ. また極大値と極小値を求めよ.
    
    \item 関数 $g(x) = e^{-x^2}$ のグラフをかけ. また最大値を求めよ.
    
\end{enumerate}
 
 問題4. 関数$f(x) = \log(1 +x)$とする. 次の問いに答えよ.
 
 \begin{enumerate}
    \setlength{\parskip}{0cm} % 段落間
  \setlength{\itemsep}{0cm} % 項目間
    \item 1階導関数$f'(x)$, 2階導関数$f^{(2)}(x)$, 3階導関数$f^{(3)}(x)$をそれぞれ求めよ.
    \item $n$階導関数$f^{(n)}(x)$を求めよ.
    \item 冪級数展開の公式$f(x) = \sum_{n=0}^{\infty}\frac{f^{n}(0)}{n!} x^{n}$であることを用いて$\log(1+x)$を冪級数展開せよ. 
\end{enumerate}


 問題5. 次の問題は2024年度の阪大理系の問題である.\footnote{ただし一部ヒントを入れて簡単にした.} 1以上の整数$n$について関数$f_{n}(x)$を
 $$
 f_n(x) = 1 - \frac{1}{2}e^{nx} +  \cos \frac{x}{3} \quad (x \geqq 0)
 $$
 で定める. 次の問いに答えよ. 
 \begin{enumerate}
   \setlength{\parskip}{0cm} % 段落間
  \setlength{\itemsep}{0cm} % 項目間
\item $f_{n}(x)$は単調減少関数であることを示せ. 
\item %$f_{n}(0)$と$\lim_{x\to +\infty} f_{n}(x)$を求めよ. また
$y=f_{n}(x)$のグラフをかくことにより
$f_{n}(x)=0$は$(0,+\infty)$上にただ一つの実数解を持つことをしめせ.
\item 2.における実数解を$a_{n}$とする. $e^{na_n} = 2 (1 + \cos \frac{a_n}{3})$であることに注意して次の不等式を示せ.
$$
0 \leqq a_n \leqq \frac{\log 4}{n}.
$$
\item $\lim_{n \to \infty} a_n$と$\lim_{n \to \infty} na_n$をそれぞれ求めよ.
\end{enumerate}
 
 \newpage
 
  \begin{center}
 {\Large 解答用紙}
% {\Large 演習問題の解答用紙 2024年1月11日(木) } \\
\end{center}

%\vspace{5pt}
%{ \large 提出締め切り 2024年月日(火) 15時10分00秒 (日本標準時刻)}
%\end{center}

%\vspace{2pt}
\begin{flushleft}
{ \large \underline{学籍番号: \hspace{4cm} 名前  \hspace{9cm}   }  }
\end{flushleft}
 

 \end{document}
