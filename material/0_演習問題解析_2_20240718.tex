\documentclass[dvipdfmx,a4paper,11pt]{article}
\usepackage[utf8]{inputenc}
%\usepackage[dvipdfmx]{hyperref} %リンクを有効にする
\usepackage{url} %同上
\usepackage{amsmath,amssymb} %もちろん
\usepackage{amsfonts,amsthm,mathtools} %もちろん
\usepackage{braket,physics} %あると便利なやつ
\usepackage{bm} %ラプラシアンで使った
\usepackage[top=20truemm,bottom=30truemm,left=25truemm,right=25truemm]{geometry} %余白設定
\usepackage{latexsym} %ごくたまに必要になる
\renewcommand{\kanjifamilydefault}{\gtdefault}
\usepackage{otf} %宗教上の理由でmin10が嫌いなので
\usepackage{showkeys}\renewcommand*{\showkeyslabelformat}[1]{\fbox{\parbox{2cm}{ \normalfont\tiny\sffamily#1\vspace{6mm}}}}
\usepackage[driverfallback=dvipdfm]{hyperref}


\usepackage[all]{xy}
\usepackage{amsthm,amsmath,amssymb,comment}
\usepackage{amsmath}    % \UTF{00E6}\UTF{0095}°\UTF{00E5}\UTF{00AD}\UTF{00A6}\UTF{00E7}\UTF{0094}¨
\usepackage{amssymb}  
\usepackage{color}
\usepackage{amscd}
\usepackage{amsthm}  
\usepackage{wrapfig}
\usepackage{comment}	
\usepackage{graphicx}
\usepackage{setspace}
\usepackage{pxrubrica}
\usepackage{enumitem}
\usepackage{mathrsfs} 

\setstretch{1.2}


\newcommand{\R}{\mathbb{R}}
\newcommand{\Z}{\mathbb{Z}}
\newcommand{\Q}{\mathbb{Q}} 
\newcommand{\N}{\mathbb{N}}
\newcommand{\C}{\mathbb{C}} 
\newcommand{\Sin}{\text{Sin}^{-1}} 
\newcommand{\Cos}{\text{Cos}^{-1}} 
\newcommand{\Tan}{\text{Tan}^{-1}} 
\newcommand{\invsin}{\text{Sin}^{-1}} 
\newcommand{\invcos}{\text{Cos}^{-1}} 
\newcommand{\invtan}{\text{Tan}^{-1}} 
\newcommand{\Area}{\text{Area}}
\newcommand{\vol}{\text{Vol}}
\newcommand{\maru}[1]{\raise0.2ex\hbox{\textcircled{\tiny{#1}}}}
\newcommand{\sgn}{{\rm sgn}}
%\newcommand{\rank}{{\rm rank}}



   %当然のようにやる.
\allowdisplaybreaks[4]
   %もちろん.
%\title{第1回. 多変数の連続写像 (岩井雅崇, 2020/10/06)}
%\author{岩井雅崇}
%\date{2020/10/06}
%ここまで今回の記事関係ない
\usepackage{tcolorbox}
\tcbuselibrary{breakable, skins, theorems}

\theoremstyle{definition}
\newtheorem{thm}{定理}
\newtheorem{lem}[thm]{補題}
\newtheorem{prop}[thm]{命題}
\newtheorem{cor}[thm]{系}
\newtheorem{claim}[thm]{主張}
\newtheorem{dfn}[thm]{定義}
\newtheorem{rem}[thm]{注意}
\newtheorem{exa}[thm]{例}
\newtheorem{conj}[thm]{予想}
\newtheorem{prob}[thm]{問題}
\newtheorem{rema}[thm]{補足}

\DeclareMathOperator{\Ric}{Ric}
\DeclareMathOperator{\Vol}{Vol}
 \newcommand{\pdrv}[2]{\frac{\partial #1}{\partial #2}}
 \newcommand{\drv}[2]{\frac{d #1}{d#2}}
  \newcommand{\ppdrv}[3]{\frac{\partial #1}{\partial #2 \partial #3}}


%ここから本文.
\begin{document}
\pagestyle{empty}




\begin{center}
{\LARGE 期末試験の情報 } \\

{2024年度春夏学期 大阪大学 全学共通教育科目 解析学入門 経(161〜)}
\end{center}

期末試験の情報は次のとおりです. 
\begin{enumerate}
  %\setlength{\parskip}{0cm} 
 % \setlength{\itemsep}{0cm}
\item 日時: 2024年 7月25日 木曜3限(13:30-15:00) \underline{13:15までにこの教室に来てください.}
\item 場所: 共B218
\item 持ち込みに関して: \underline{A4用紙4枚(裏表使用可)のみ持ち込み可.} 工夫を凝らしてA4用紙4枚に今までの内容をまとめてください. (A4用紙はこの用紙のサイズです.) A4より大きいサイズの紙を用いた場合, その用紙を没収します. その他(教科書, スマートフォン, 携帯)は使用できません.
\item 試験内容: 授業でやった範囲
\end{enumerate}

以下は注意事項です.
\begin{itemize}
  \setlength{\parskip}{0cm} 
 \setlength{\itemsep}{0cm}
 
 \item \underline{解答に関して, 答えのみならず, 答えを導出する過程をきちんと記してください.} きちんと記していない場合は大幅に減点する場合がある.
 \item 期末試験には「普通の問題」と「おまけの問題」があります. 普通の問題はしっかり勉強すれば解ける問題です. おまけの問題は解けることを想定していない問題です. 面白いので出しました.  

\item 途中退出は14:00-14:45までとします. 試験が早く解けたものや諦めたものはこの時間に試験を提出し, その後退出してください. 
%\item 試験問題はちょっと難しめに設定しております. これは今回持ち込み可にしたことと成績の差をつけたいからです. 
\item 何をやればいいかわからない人は, \underline{最低限として演習問題を解けるようにしてください.} (ただしそれだけで単位が来るとは限らないです.) また単位を認定するくらいの成績が取れていない場合, 容赦無く不可を出します.

\item 試験対策として作ったA4用紙4枚は試験後も捨てずに置いておくことをお勧めします. なぜならこの用紙4枚にこの授業で学ぶべき内容が詰まっているからです. 
\end{itemize}


演習問題及び授業の資料・板書内容は授業ページ(\url{https://masataka123.github.io/2024_summer_calculus/})にもあります. 
下のQRコードからを読み込んでも構いません.
 \vspace{11pt}
\begin{figure}[h]
  \centering
 \includegraphics[height=30mm, width=30mm]{calculus.png}
\end{figure}




\newpage 

%\begin{center}
%{\Large 2023年度 幾何学基礎2(位相空間論)演義 期末レポート} \\

%\vspace{5pt}
%{ \large 提出締め切り 2024年月日(火) 15時10分00秒 (日本標準時刻)}
%\end{center}

\begin{center}
{\Large 演習問題  2024年7月18日(木) } 
\end{center}

%\vspace{5pt}
%{ \large 提出締め切り 2024年月日(火) 15時10分00秒 (日本標準時刻)}
%\end{center}

%\vspace{2pt}
%%\begin{flushleft}
%{ \large \underline{学籍番号: \hspace{4cm} 名前  \hspace{8cm} } }
%\end{flushleft}


\begin{center}
 {\large 下の問題を解け. なお解答は配布した解答用紙に解答すること.}
  \end{center}
 ただし解答に関しては答えのみならず, 答えを導出する過程をきちんと記すこと. 
 また解答用紙は1人1枚以上提出すること.
 
  \vspace{11pt}
  問題1.  次の定積分の値を求めよ. 
  \begin{enumerate}
 \item   $\int_0^{\pi/2} (\cos x)^2 \, dx$ (ヒント: 2倍角の公式$\cos2x = 2(\cos x)^2 -1$を使う)
 \item $ \int_0^{\pi} (\sin x)^2 \cos x \, dx$ (ヒント: $t^3$の微分.)
 \item   $\int_1^2 (\log x)^2 \, dx $ (ヒント: 部分積分法)
 \item$ \int_{-1}^{1} \frac{1}{1 + x^2} \, dx$ (ヒント: 置換積分法 $x = \tan t$)
  \end{enumerate}

  \vspace{5pt}
  問題2. 次の問い答えよ. 

  \begin{enumerate}
  %  \setlength{\parskip}{0cm} 
 %\setlength{\itemsep}{0cm}  
   %\item 関数 $f(x,y) = x^2 + 3xy - 2y^2$ の各偏導関数$\pdrv{f}{x}, \pdrv{f}{y}$を求めよ
 
     \item 関数 $f(x,y) = \frac{x}{y} + \log x$の各偏導関数$\pdrv{f}{x}, \pdrv{f}{y}$を求めよ.
        
    \item 関数 $f(x,y) = e^{xy}\sin x$ の各偏導関数$\pdrv{f}{x}, \pdrv{f}{y}$を求めよ.
    
  
  
    \end{enumerate}
    
       \vspace{11pt}
    問題3. 
  $\R^2$上の$C^\infty$級関数を
  $$f(x,y) = x^2 + xy+2y^2-4y$$
  とする.
$f(x,y) $について極大点・極小点を持つ点があれば, その座標と極値を求めよ. またその極値が極大値か極小値のどちらであるか示せ.  

   \vspace{11pt}
問題4.

 $$f(x,y) = x^2+xy+y^2,\quad  g(x,y) = x^2+y^2-1$$
 とする.
$g(x,y) =0$のもとでの$f(x,y)$の最大値と最大値をとる点の座標, 最小値と最小値をとる点の座標を全て求めよ. 

\vspace{4pt}
つまり$S = \{ (x,y) \in \R^2: g(x,y)=0\}$とするとき, $f$の$S$上での最大値と最大値をとる点の座標, 及び最小値と最小値をとる点の座標を全て求めよ.
ただし$S$上で$f(x,y)$が最大値・最小値をとることは認めて良い.
 \newpage

 \begin{center}
 {\Large 解答用紙}
% {\Large 演習問題の解答用紙 2024年1月11日(木) } \\
\end{center}

%\vspace{5pt}
%{ \large 提出締め切り 2024年月日(火) 15時10分00秒 (日本標準時刻)}
%\end{center}

%\vspace{2pt}
\begin{flushleft}
{ \large \underline{学籍番号: \hspace{4cm} 名前  \hspace{9cm}   }  }
\end{flushleft}


 \end{document}
